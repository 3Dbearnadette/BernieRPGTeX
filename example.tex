% Changing book to article will make the footers match on each page,
% rather than alternate every other.
%
% Note that the article class does not have chapters.
\documentclass[10pt,twoside,twocolumn,openany]{book}

% Use babel or polyglossia to automatically redefine macros for terms
% Armor Class, Level, etc...
% Default output is in English; captions are located in lib/dndstring-captions.sty.
% If no captions exist for a language, English will be used.
%1. To load a language with babel:
%	\usepackage[<lang>]{babel}
%2. To load a language with polyglossia:
%	\usepackage{polyglossia}
%	\setdefaultlanguage{<lang>}
\usepackage[english]{babel}
%usepackage[italian]{babel}
% For further options (multilanguage documents, hypenations, language environments...)
% please refer to babel/polyglossia's documentation.

\usepackage[utf8]{inputenc}
\usepackage{lipsum}
\usepackage{listings}
\usepackage[bg-letter]{dnd} % Options: bg-a4, bg-letter, bg-full, bg-print, bg-none, justified

\lstset{%
  basicstyle=\ttfamily,
  language=[LaTeX]{TeX},
}

% Start document
\begin{document}

% Your content goes here

% Comment this out if you're using the article class.
\chapter{Chapter 1: The Dark \LaTeX}

\section{Main Section}
\lipsum[2] % filler text

\begin{quotebox}
	As you approach this template you get a sense that the blood and tears of many generations went into its making. A warm feeling welcomes you as you type your first words.
\end{quotebox}

\subsection{Fun with boxes}
\subsubsection{Even more fun!}

\begin{commentbox}{This Is a Comment Box!}
  A \lstinline!commentbox! is a box for minimal highlighting of text. It lacks the ornamentation of \lstinline!paperbox!, but it can handle being broken over a column.
\end{commentbox}

\subtitlesection{Weapon, +1, +2, or +3}
{Weapon (any), uncommon (+1), rare (+2), or very rare (+3)}

\lipsum[3]

\begin{figure}[!t]
	\begin{paperbox}{Behold\, the Paperbox!}
		The \lstinline!paperbox! is used as a sidebar. It does not break over columns and is best used with a figure environment to float it to one corner of the page where the surrounding text can then flow around it.
	\end{paperbox}
\end{figure}

% For more columns, you can say \begin{dndtable}[your options here].
% For instance, if you wanted three columns, you could say
% \begin{dndtable}[XXX]. The usual host of tabular parameters are
% available as well.
\header{Nice table}
\begin{dndtable}
   	\textbf{Table head}  & \textbf{Table head} \\
   	Some value  & Some value \\
   	Some value  & Some value \\
   	Some value  & Some value
\end{dndtable}

% You can optionally not include the background by saying
% begin{monsterboxnobg}
\begin{monsterbox}{Monster Foo}
	\textit{Small metasyntactic variable (goblinoid), neutral evil}\\
	\hline%
	\basics[%
	armorclass = 12,
	hitpoints  = \dice{3d8 + 3},
	speed      = 50 ft
	]
	\hline%
	\stats[
    STR = \stat{12}, % This stat command will autocomplete the modifier for you
    DEX = \stat{7}
	]
	\hline%
	\details[%
	% If you want to use commas in these sections, enclose the
	% description in braces.
	% I'm so sorry.
	languages = {Common Lisp, Erlang},
	]
	\hline \\[1mm]
	\begin{monsteraction}[Monster-super-powers]
		This Monster has some serious superpowers!
	\end{monsteraction}
	\monstersection{Actions}
	\begin{monsteraction}[Generate text]
		This one can generate tremendous amounts of text! Though only when it wants to.
	\end{monsteraction}

	\begin{monsteraction}[More actions]
    See, here he goes again! Yet more text.
	\end{monsteraction}
\end{monsterbox}

\section{Spells}

\begin{spell}
	{Beautiful Typesetting}
	{4th-level illusion}
	{1 action}
	{5 feet}
	{S, M (ink and parchment, which the spell consumes)}
	{Until dispelled}
	You are able to transform a written message of any length into a beautiful scroll. All creatures within range that can see the scroll must make a wisdom saving throw or be charmed by you until the spell ends.

	While the creature is charmed by you, they cannot take their eyes off the scroll and cannot willingly move away from the scroll. Also, the targets can make a wisdom saving throw at the end of each of their turns. On a success, they are no longer charmed.
\end{spell}

\lipsum[2]

\section{Colors}

This package provides several global color variables to style \lstinline!commentbox!, \lstinline!quotebox!, \lstinline!paperbox!, and \lstinline!dndtable! environments.

\begin{dndtable}[lX]
  \textbf{Color}         & \textbf{Description} \\
  \lstinline!commentboxcolor! & Controls \lstinline!commentbox! background. \\
  \lstinline!paperboxcolor!   & Controls \lstinline!paperbox! background. \\
  \lstinline!quoteboxcolor!   & Controls \lstinline!quotebox! background. \\
  \lstinline!tablecolor!      & Controls background of even \lstinline!dndtable! rows. \\
\end{dndtable}

See Table~\ref{tab:colors} for a list of accent colors that match the core books.

\begin{table*}
  \begin{dndtable}[XX]
    \textbf{Color}                            & \textbf{Description} \\
    \lstinline!PhbLightGreen!                      & Light green used in PHB Part 1 \\
    \lstinline!PhbLightCyan!                       & Light cyan used in PHB Part 2 \\
    \lstinline!PhbMauve!                           & Pale purple used in PHB Part 3 \\
    \lstinline!PhbTan!                             & Light brown used in PHB appendix \\
    \lstinline!DmgLavender!                        & Pale purple used in DMG Part 1 \\
    \lstinline!DmgCoral!                           & Orange-pink used in DMG Part 2 \\
    \lstinline!DmgSlateGray! (\lstinline!DmgSlateGrey!) & Blue-gray used in PHB Part 3 \\
    \lstinline!DmgLilac!                           & Purple-gray used in DMG appendix \\
  \end{dndtable}
  \caption{Colors supported by this package}%
  \label{tab:colors}
\end{table*}

\begin{itemize}
  \item Use \lstinline!\setthemecolor[<color>]! to set \lstinline!themecolor!, \lstinline!commentcolor!, \lstinline!paperboxcolor!, and \lstinline!tablecolor! to a specific color.
  \item Calling \lstinline!\setthemecolor! without an argument sets those colors to the current \lstinline!themecolor!.
  \item \lstinline!commentbox!, \lstinline!dndtable!, \lstinline!paperbox!, and \lstinline!quoteboxcolor! also accept an optional color argument to set the color for a single instance.
\end{itemize}

\subsection{Examples}

\subsubsection{Using \lstinline!themecolor!}

\begin{lstlisting}
\setthemecolor[PhbMauve]

\begin{paperbox}{Example}
  \lipsum[2]
\end{paperbox}

\setthemecolor[PhbLightCyan]

\header{Example}
\begin{dndtable}[cX]
  \textbf{d8} & \textbf{Item} \\
  1           & Small wooden button \\
  2           & Red feather \\
  3           & Human tooth \\
  4           & Vial of green liquid \\
  6           & Tasty biscuit \\
  7           & Broken axe handle \\
  8           & Tarnished silver locket \\
\end{dndtable}
\end{lstlisting}

\begingroup
\setthemecolor[PhbMauve]

\begin{paperbox}{Example}
  \lipsum[2]
\end{paperbox}

\setthemecolor[PhbLightCyan]

\header{Example}
\begin{dndtable}[cX]
  \textbf{d8} & \textbf{Item} \\
  1           & Small wooden button \\
  2           & Red feather \\
  3           & Human tooth \\
  4           & Vial of green liquid \\
  6           & Tasty biscuit \\
  7           & Broken axe handle \\
  8           & Tarnished silver locket \\
\end{dndtable}
\endgroup

\subsubsection{Using element color arguments}

\begin{lstlisting}
\begin{dndtable}[cX][DmgCoral]
  \textbf{d8} & \textbf{Item} \\
  1           & Small wooden button \\
  2           & Red feather \\
  3           & Human tooth \\
  4           & Vial of green liquid \\
  6           & Tasty biscuit \\
  7           & Broken axe handle \\
  8           & Tarnished silver locket \\
\end{dndtable}
\end{lstlisting}

\begin{dndtable}[cX][DmgCoral]
  \textbf{d8} & \textbf{Item} \\
  1           & Small wooden button \\
  2           & Red feather \\
  3           & Human tooth \\
  4           & Vial of green liquid \\
  6           & Tasty biscuit \\
  7           & Broken axe handle \\
  8           & Tarnished silver locket \\
\end{dndtable}

% End document
\end{document}
